\documentclass[a4paper,11pt,twoside]{book}
%\documentclass[11pt,twoside,letterpaper]{book}
\usepackage{a4wide}
%\usepackage{multirow}
\usepackage{footnote}
\usepackage{amsbsy}
\usepackage[dvips]{graphicx}
%\usepackage{fancyheadings}
\usepackage{fancyhdr}
%\setlength{\parindent}{0in}
%\setlength{\parskip}{0.05in}
\setlength{\parskip}{0.1in}
\bibliographystyle{plain}
%\usepackage{a4wide}
\usepackage{color}
\def\red{\textcolor{red}}
\usepackage{lscape}

%\parskip=2mm
\newcommand{\kbf}{\mathbf{k}}
\renewcommand{\d}{\mathrm{d}}
\newcommand{\e}{\mathrm{e}}

% Ensure that blank pages don't have numbers or heading on them 
\makeatletter
\def\cleardoublepage{\clearpage\if@twoside \ifodd\c@page\else
 \hbox{}
 \vspace*{\fill}
 \thispagestyle{empty}
 \newpage\fi\fi}
\makeatother

% set fancy headings
\pagestyle{fancy}
\lhead[{\it \thepage}]{{\bf\it {\tt OptaDOS}: User Guide}}
\chead{}
\rhead[{\bf\it {\tt OptaDOS}: User Guide}]{{\it \thepage}}
\renewcommand{\headrulewidth}{0.2pt}
\lfoot{}
\cfoot{}
\rfoot{}
\renewcommand{\footrulewidth}{0pt}
\setlength{\footskip}{0.25in}
\setlength{\parindent}{0in}



\title{{\huge {\tt OptaDOS}: User Guide}\\ {Version 1.1.372}}

\author{Andrew J. Morris, Rebecca J. Nicholls, Chris J. Pickard, Jonathan R. Yates \\
\\
Department of Physics\\
University of Cambridge\\
19 J.~J. Thomson Ave\\
Cambridge CB3 0FF\\
UK,\\
\\
Department of Materials\\
University of Oxford\\
Parks Road\\
Oxford OX1 3PH\\
UK \\
\\
{\small and} \\
\\
Department of Physics and Astronomy\\
University College London\\
Gower Street\\
London WC1E 6BT\\
UK}


\date{March 2014}

\begin{document}
\newcommand{\optados}{\textsc{OptaDOS}}
\newcommand{\lindos}{\texttt{LinDOS}}
\newcommand{\onetep}{\textsc{onetep}}
\newcommand{\castep}{\textsc{castep}}
\maketitle

%%%%% Copyright page
 \thispagestyle{empty}

\begin{centering}
\vspace*{40mm}
University of Cambridge, University of Oxford and University College London\\
\vspace{5mm}
\copyright\ 2013-2014 A.~J. Morris, R.~J. Nicholls, C.~J. Pickard and J.~R. Yates.\\
\vspace{5mm}
First published 2010\\
This edition 2014\\
\vspace{5mm}
10 9 8 7 6 5 4 3 2 1
\vspace{5mm}

Part of this work published in:\\
``OptaDOS: A Tool for Obtaining Density of States, Core-loss and Optical Spectra from Electronic Structure Codes'', Andrew J. Morris, Rebecca J. Nicholls, Chris J. Pickard and Jonathan R. Yates, Comp. Phys. Comm. {\bf 185}, 5, 1477 (2014)\\
\vspace{10mm}
Further information about \optados\ may be obtained from:\\ \texttt{www.optados.org}\\

\vspace{10mm}
\emph{Nota bene:} \optados, optados or OptaDOS, but never OPTADOS.

\null\vfill
\noindent
Font: Computer Modern, 11pt\\
Typeset by the authors using \LaTeX\, \\
\end{centering}
\newpage

\setcounter{tocdepth}{1}
\tableofcontents

\newpage
 \thispagestyle{empty}
\vspace*{50mm}
\begin{centering}
We would like to thank Phil Hasnip and Keith Refson for very helpful discussions and Gareth Griffiths and Kane O'Donnell for extensive beta testing.
 AJM, RJN and CJP acknowledge support from the EPSRC. Additionally AJM acknowledges
the Winton programme for the physics of sustainability and RJN the ERC
(Grant ERC-2009-StG-240500).
 JRY acknowledges support from the Royal Society through a University Research fellowship. 
\end{centering}
\begin{flushright}
AJM, RJN, CJP and JRY\\
September 2013
\end{flushright}
\newpage


\chapter{Introduction}\label{chap:introduction}
\section{Background}
\optados\ is a code for calculating optical, core-level excitation spectra along with full, partial and joint electronic density of states (DOS).  The code was developed by merging the \lindos\ code of Andrew Morris and Chris Pickard at University College London with the optical properties code of Rebecca Nicholls and Jonathan Yates at Oxford University.  \optados\ is written in Fortran 95 and may be run in parallel using {\tt MPI}.  At present \optados\ interfaces with \castep\ and \onetep\ output files, although it is extendable to perform calculations on any set of band eigenvalues and their derivatives generated by any electronic structure code.

The code is freely available through the GPL licence with the request that the following citation (quoted in full) is required in any publication resulting from the use of \optados.

\emph{Andrew\,J. Morris, R.\,J. Nicholls, C.\,J. Pickard and Jonathan\,R. Yates, \optados: A Tool for Obtaining Density of States, Core-level and Optical Spectra from Electronic Structure Codes,  Comp. Phys. Comm. {\bf 185}, 5, 1477,(2014)}

\begin{center}
Further information and examples can be found at

\verb#www.optados.org#
\end{center}


\section{Features}
\optados\ generates optical, core-level excitation spectra along with full, partial and joint electronic DOS. The DOS, PDOS and JDOS take advantage of the linear and adaptive broadening schemes which are more accurate than standard Gaussian broadening since they exploit knowledge of the gradients of the bands at each $k$-point in the Brillouin zone.  These DOS are the basis of the more advanced functionality of \optados, the core and optical spectra.

Along with data text files \optados\ also generates \verb#.agr# files of results to be read by \verb#grace#. 




%%%%%%%%%%%%%%%%%%%%%%%%%%%%%%%%
\chapter{Theoretical Background} \label{sec:theory}
%%%%%%%%%%%%%%%%%%%%%%%%%%%%%%%%

\section{Integrals over the Brillouin Zone}

Many energy-resolved spectral properties of a material take the form of an integration of some function, $F_{nm}(\kbf)$ in  reciprocal space over the BZ; where $F_{nm}(\kbf)$ are the elements of a periodic operator which commutes with the crystal translational symmetry.
%
Perhaps the two simplest examples of integrals, hereinafter referred to as type (a) and type (b), take the form at T$=0\,$K of,
\begin{equation}
I^{(a)} \left ( E \right) = \sum_n \int_{\rm BZ} \frac {\d\kbf}{(2\pi)^3} F_{nn}(\kbf) \delta (E_{n\kbf}-E),
\end{equation}
and,
\begin{equation}
I^{(b)} \left ( E \right) = \sum^{\rm occ}_n \sum^{\rm unocc}_m
\int_{\rm BZ} \frac {\d\kbf}{(2\pi)^3} F_{nm}(\kbf) \delta ( E - (E_{m\kbf}-E_{n\kbf})),
\label{Eqn:IntB}
\end{equation}
where $E_{n\kbf}$ are the energy eigenvalues of the system. 
%
They are also extendable to integration over spin-polarised channels
in a straightforward way.

\section{Brillouin Zone Sampling Schemes}

Since the crystal momentum, $\kbf$, is a continuous quantum number the band structure in an \emph{ab initio} calculation is approximated using specific $k$-points within the first Brillouin zone.
%
These $k$-points may be chosen through symmetry perhaps using a Monkhorst-Pack grid \cite{monkhorst:PRB:1976} or  the multi-$k$-point generalisation to the Baldereschi scheme\cite{rajagopal:PRL:1994,morris:PRB:2008}.


Obtaining the energy bands from a set of eigenenergies at discrete $k$-points is non-trivial. 
%
This simplest way of assigning bands, by counting eigenvalues at $k$-points, implicitly assumes that there are no band crossings in the BZ.
%
This is further complicated by ``band kissing'' where bands approach, but due to small interactions between the bands breaking the degeneracy they remain continuous\cite{blochl:PRB:1994,pickard:PRB:1999}.
%
Considering band coefficients and the overlap matrix it has been shown how to assign bands to individual eigenenergies\cite{yazyev:PRB:2002,refson:o2b}, however these still require a broadening scheme to obtain a DOS.
%
Below we detail some approaches to obtaining a DOS from a set of eigenenergies.



\subsection{Gaussian broadening}
%
The reciprocal space is discretised into \emph{sub-cells} each containing a $k$-point which contributes to the integral corresponding to the energy of the $k$-point at their centre.
%
Band dispersion may be crudely approximated by smearing each sub-cell's contribution by a fixed-width Gaussian function, of some width, $\omega$,
\begin{equation}
\delta (E_{n\kbf}-E) \rightarrow \frac{1}{\omega\sqrt{2\pi}}\exp \left (-\frac{1}{2}((E_{n\kbf}-E)/\omega)^2 \right ).  
\end{equation}
This approach requires a large number of $k$-points to converge the DOS since when choosing $\omega$ there is a trade-off between representing the sharp features, such as van Hove singularities\cite{vanHove:PR:1953}  and not introducing spurious oscillations due to the limited $k$-point sampling.
%
The band crossing problem is not present since the bands are modelled as dispersionless.

Fixed-width Gaussian broadening may be chosen  for \optados\ calculations with, \texttt{broadening : fixed}.


\subsection{Interpolative schemes}
In the linear tetrahedron interpolative scheme the BZ is divided into
roughly equal size tetrahedra with the band energy calculated at each
vertex \cite{lehmann:PSS:1972}.
%
The DOS contribution of each tetrahedron is then calculated analytically by linear interpolation.

Linear interpolation suffers from the band crossing problem, since the interpolation always joins the lowest energy bands together, which has the effect of avoiding all band crossing and kissing.
%
The method also does not describe van Hove singularities well since the interpolation is linear. 

Higher-order interpolation schemes, such as the quadratic tetrahedron scheme,  have been proposed which converge the singularities more rapidly than the linear scheme  \cite{boon:JPC:1986,methfessel:JPC:1983,methfessel:JPC:1987}, but these still require a high-density $k$-point grid to mitigate the band-crossing problem.


\subsection{Extrapolative schemes}
Extrapolating from each $k$-point rather than interpolating between adjacent points eliminates the band-crossing problem.
%
M\"uller and Wilkins show that for free electron bands the DOS is a smooth parabola when using the extrapolative method and a rough unconverged spectrum with the interpolative approach when using the same $k$-point mesh \cite{muller:PRB:1984}.

The linear extrapolative scheme still does not represent van Hove singularities as well as a second-order approach.  
%
Band curvatures may be combined with L\"owdin perturbation theory to extrapolate to higher order \cite{pickard:PRB:1999,pickard:PRB:2000}.

Linear extrapolative broadening may be chosen  for \optados\ calculations with, \texttt{broadening : linear}.


\subsection{Adaptive broadening schemes}
An approximation to linear extrapolation method was proposed by Yates \emph{et al.} \cite{yates:PRB:2007}.
%
Each sub-cell's contribution to the DOS is broadened by a Gaussian function whose width, $\omega$, is proportional to the energy band gradient as its $k$-point.
%
This simple approach removes the spurious oscillations of the fixed broadening whilst maintaining the sharper features.
%
The adaptive broadening scheme was introduced in Ref \cite{yates:PRB:2007} in the context of Wannier interpolation of BZ quantities. 
%
However, the scheme is more general that this, and with \optados\ we apply adaptive broadening directly to the quantities obtained from the electronic structure calculation.

Adaptive broadening may be chosen  for \optados\ calculations with, \texttt{broadening : adaptive}.



%%%%%%%%%%%%%%%%%%%%%%%%%%%%%%%%
\section{Density of Electronic States} 
%%%%%%%%%%%%%%%%%%%%%%%%%%%%%%%%

When $F_{nm}(\kbf)=1$ the type (a) integral represents the electronic density of states (DOS) at $E$ and the total ground-state energy, $E_{\rm gs}$ is obtained by integrating over the occupied energies,
\begin{equation}
E_{\rm gs} = \int^{E_{F}}_0 I^{(a)}( E; F_{nm}(\kbf)=1) \d E,
\end{equation}
where $E_F$ is the Fermi energy.

The DOS is plotted with \optados\ using the \verb@task : dos@ command.

The partial, or projected DOS (PDOS), $I_{\mu}$ is obtained by projecting the DOS onto local orbitals, $\mu$, at each atomic site\cite{segall:MP:1996}.
%
In this case,
\begin{equation}
F_{\mu n}(\kbf)=\Re \left \{  T_{\mu n}^{\dagger}(\kbf) T_{\nu n}(\kbf) S_{\nu \mu}(\kbf)^{-1} \right \},
\end{equation}
where $ T_{\mu n}(\kbf)= \langle \Psi_n(\kbf)|\phi_{\mu}(\kbf)
\rangle$ are the overlap matrices between LCAO (linear combination of
atomic orbitals) basis, $\phi_{\mu}$ and planewave states, $\Psi_n$, and  $ S_{\mu n}(\kbf)=\langle \phi_n(\kbf)|\phi_{\mu}(\kbf) \rangle$, the overlap matrices of the LCAO basis.

The PDOS is plotted with \optados\ using the \verb@task : pdos@
command. 
%
The projection is defined using the \verb@pdos@ keyword as described
in the user manual.

In the case of the type (b) integral when $F_{nm}(\kbf)=1$, we
obtain the joint-DOS, (JDOS), for frequency, $\omega$.
%
The JDOS is plotted with \optados\ using the \verb@task : jdos@ command.



\section{Dielectric response}

Core-level spectra and optical properties are related to the complex dielectric function $\varepsilon_1+i\varepsilon_2$.  Within the random phase approximation, the imaginary part of the dielectric function is
\begin{equation}
\varepsilon_2 \left ( \omega \right ) = \frac{\pi e^2}{\varepsilon_0 \Omega} \frac{1}{q^2} \left | \langle  \psi^c_{\bf k}   |  \e^{-i{\bf q}.{\bf r}} | \psi^v_{\bf k + q} \rangle\right |^2,
\end{equation}
where $e$ is the charge on an electron, $\varepsilon_0$ is the permittivity of free space, $\Omega$ is the volume of the unit cell and {\bf q} is the momentum transfer \cite{dressel}.
Using the dipole approximation, expressing {\bf q} as $q\,\hat{\bf{u}}$ and taking the limit that $q \rightarrow 0$, $\varepsilon_2$ can be written as an integral of type (b), 
\begin{equation}
\varepsilon_2 \left ( \omega \right ) = \frac{\pi e^2}{\varepsilon_0 \Omega} I^{(b)}\left ( \hbar \omega \right ), 
\label{Eqn:Epsilon2}
\end{equation}
where,
\begin{equation}
F_{nm}(\kbf)= \left | \langle \psi^c_{\bf k}  |  \hat{\bf u}\cdot {\bf r} | \psi^v_{\bf k} \rangle\right |^2.
\label{Eqn:Dipole}
\end{equation}
Once $\varepsilon_2$ is known, it is possible to use the Kramers-Kronig relations to obtain $\varepsilon_1$ and hence obtain an expression for the full dielectric function.  The dielectric function is a tensor and in \optados\ it is either possible to output the full dielectric tensor or calculate dielectric function for the isotropic average of {\bf q}, a particular direction of {\bf q} or the average of {\bf q} over a plane perpendicular to a given direction.  Once the dielectric function has been calculated, core-level spectra and optical properties can be calculated.  

Core-level and low-loss EELS spectra are all related to the loss function, which is defined as \cite{egerton}:
\begin{equation}
\Lambda = \Im \left ( \frac{-1}{\varepsilon_1 + i\varepsilon_2  } \right ).
\label{Eqn:LossFn}
\end{equation}


\subsection{Core-level spectra}

For core-level spectroscopy (core-loss EELS and x-ray absorption near edge spectroscopy (XANES)) the incident perturbation is far from resonance and since $\varepsilon_1\approx1$ and $\varepsilon_2\gg1$ equation\,{\ref{Eqn:LossFn} reduces to  $\Lambda = \varepsilon_2$ \cite{egerton}.  
The type (b) integral in equation \ref{Eqn:Epsilon2} becomes a type (a) integral as the energy of the core state is fixed.  Emission spectra (transitions from valence to core XAS) can also be calculated by considering a sum over the occupied rather than unoccupied states.  Core-level spectra are calculated in \optados\ using the \verb@task : core@ command.

Several different experimental geometries are possible for core-level experiments and \optados\ allows the calculation of spectra with {\bf q} in a particular direction or an isotropic average.  
In core-level absorption spectra there are several sources of broadening coming from the experimental set up and lifetime effects.  These can be included in \optados\ by broadening the theoretical spectrum using a combination of Gaussian and Lorentzian functions.  To include instrumentation and lifetime effects, \verb@core_LAI_broadening@ should be set to true.  

\subsection{Optical properties}

In the low-loss EELS regime, the approximations used for core-level spectroscopy do not hold and the full form of the loss function needs to be calculated.  This is done by calculating $\varepsilon_2$ using equations \ref{Eqn:Epsilon2} and \ref{Eqn:Dipole} and then using the Kramers-Kronig relations to find $\varepsilon_1$.  Once the dielectric function has been calculated, the loss-function is simulated (without local field effects) using equation \ref{Eqn:LossFn}. As the dielectric function has been calculated, several other optical properties (which are listed below) can also be computed \cite{dressel}. 

Conductivity, in units of Sm$^{-1}$, is computed using:
\begin{equation}
\sigma_1 = \varepsilon_0 \varepsilon_2 \omega, 
\end{equation}
\begin{equation}
\sigma_2 = \varepsilon_0 \left ( 1 - \varepsilon_1 \right ) \omega.
\end{equation}

The refractive index is calculated using:
\begin{equation}
N=n+ik
\end{equation}
where $n$ and $k$ are obtained from,
\begin{eqnarray}
n^2=\frac{1}{2}([\varepsilon_1^2+\varepsilon_2^2]^{\frac{1}{2}}+\varepsilon_1)
& {\rm \;and\;}&
k^2=\frac{1}{2}([\varepsilon_1^2+\varepsilon_2^2]^{\frac{1}{2}}-\varepsilon_1).
\end{eqnarray}


The absorption coefficient, (in units of m$^{-1}$), is computed using
\begin{equation}
\eta = \frac{2k\omega}{c},
\end{equation}
and reflection coefficient, $R$, from,
\begin{equation}
R= \frac{(n-1)^2 + k^2}{(n+1)^2+k^2}.
\end{equation}

All optical properties are calculated in \optados\ using the \verb@task : optics@ command.

\subsection{Intraband term}
For metals it is necessary to consider an intraband contribution to
the optical response, i.e. a term when $n = m$ in equation
\ref{Eqn:IntB}  \cite{dressel,draxl}.  The contribution to
$\varepsilon$ can be written as a type (a) integral with $\hbar\omega$
set to $E_F$, 
\begin{equation}
\varepsilon_1^{\rm{intra}} \left ( \omega \right ) = 1 - \frac{e^2}{\varepsilon_0 \Omega} \frac{1}{(\omega^2+\Gamma^2)} I^{(a)}\left ( E_F \right ),
\end{equation}
\begin{equation}
\varepsilon_2^{\rm{intra}} \left ( \omega \right ) = \frac{e^2}{\varepsilon_0 \Omega}\frac{\Gamma}{\omega (\omega^2+\Gamma^2)} I^{(a)}\left ( E_F \right ),
\end{equation}
where $\Gamma$ denotes the relaxation rate.  



\chapter{Getting Started}\label{chap:getting_started}
\section{Installation}
\optados\ is usually obtained in a gzipped tarball, \verb#optados-X.X.tar.gz#. Extract this ( \verb# tar -xzf optados-X.X.tar.gz#) in the desired directory. Inside the  \verb#optados/# directory are a number of sub directories,  \verb#documents/#,   \verb#examples/# and \verb#tools/#.  The code may be compiled using the \verb#Makefile# in the  \verb#optados/# directory.  The \verb#SYSTEM#, \verb#BUILD#, \verb#COMMS_ARCH# and \verb#PREFIX# flags must be set, either in the  \verb#makefile.system#, or from the command line (for example  \verb#make BUILD=fast#).  

\subsection[system]{\tt SYSTEM}

Choose which compiler to use to make \optados. The valid values are:
\begin{itemize}
\item[{\bf --}]  \verb#g95# (default)
\item[{\bf --}]  \verb#gfortran#
\item[{\bf --}]  \verb#ifort#
\item[{\bf --}]  \verb#nag#
\item[{\bf --}]  \verb#pathscale#
\item[{\bf --}]  \verb#pgf90#
\item[{\bf --}]  \verb#sun#
\end{itemize}

\subsection[build]{\tt BUILD}

Choose the level of optimisations required when making \optados.  The valid values are:
\begin{itemize}
\item[{\bf --}]  \verb#fast# (default) All optimisations
\item[{\bf --}]  \verb#debug# No optimisations, full debug information
\end{itemize}

\subsection[comms_arch]{\tt COMMS\_ARCH}

Whether to compile for serial or parallel execution. The valid values are:
\begin{itemize}
\item[{\bf --}]  \verb#serial# (default) 
\item[{\bf --}]  \verb#mpi# 
\end{itemize}

\subsection[bin_dir]{\tt PREFIX}
Choose where to place the \optados\ binary. The default is the \optados\ directory.


\section{Usage}
{\tt
\begin{quote}
optados.x86\_64  [seedname]
\end{quote} }
\begin{itemize}
\item{  {\tt seedname}: If a seedname string is given the code will read its input from a file {\tt seedname.odi}. The default value is \castep.}
\end{itemize}


%%%%%%%%%%%%%%%%%%%%%%%%%%%%%%%%%%
\chapter{Structure of the Program} \label{sec:structure}
%%%%%%%%%%%%%%%%%%%%%%%%%%%%%%%%%
The schematic structure of the program is outlined in Fig.~\ref{fig:prog_structure}. 
%
Each rounded-cornered box represents a Fortran95 module and the lines represent module dependencies. 
%
Modules only use data and subroutines from lower modules in the diagram. 
%
The modules are grouped into functional, structural, utility and low level.
%
Low level modules set up the environment for communication in serial or parallel between nodes and to disk. 
%
The utility modules are not context dependent on the problem, and contain general algorithms and an interface to write \texttt{xmgrace} input files.
%
Structural modules define data structures, hold data and perform low-level operations on them.
%
The functional modules perform the higher-level data operations and generate output.

%
A description of the purpose of each module is given below.


\begin{itemize}
\item \texttt{algorithms}: subroutines not specific to electronic structure
\item \texttt{cell}: real and reciprocal space data manipulation and storage, (\emph{e.g.} cell vectors and $k$-points)
\item \texttt{comms}: interfaces to parallel libraries
\item \texttt{constants}: definition of constants and conversion factors
\item \texttt{core}: perform core-level calculation
\item \texttt{dos}: perform DOS calculation
\item \texttt{dos\_utils}: perform type (a) integral evaluation
\item \texttt{electronic}: read electronic eigenvalue data. Store electronic data variables
\item \texttt{io}: error handling, timing, and input and output units
\item \texttt{jdos}: perform JDOS calculation
\item \texttt{jdos\_utils}: perform type (b) integral evaluation
\item \texttt{optados}: main program
\item \texttt{optics}: calculate dielectric function then generate optical properties
\item \texttt{parameters}: read, store and check input file parameters
\item \texttt{pdos}: perform PDOS calculation
\item \texttt{xmgrace\_utils}: routines to write out data in xmgr format
\end{itemize}

\begin{figure}
\begin{center}
\includegraphics*[width=100mm]{./images/block_diag.eps}
\caption{Schematic structure of the program} \label{fig:prog_structure}
\end{center}
\end{figure}

\chapter{Parameters}\label{chap:parameters}

\section{{\tt seedname.odi} File}
The \optados\ input file {\tt seedname.odi} has a flexible free-form
structure. 

The ordering of the keywords is not significant. Case is ignored (so
\verb#smearing_width# is the same as \verb#Smearing_Width#). Characters after !, or \#
are treated as comments. Most keywords have a default value that is
used unless the keyword is given in {\tt seedname.odi}. Keywords may be set
in any of the following ways
{\tt
\begin{quote}
smearing\_width = 0.4

smearing\_width : 0.4

smearing\_width   0.4
\end{quote} }
A logical keyword can be set to {\tt .true.} using any of the following
strings: {\tt T}, {\tt true}, {\tt .true.}.


\clearpage


\section{General Parameters}
\subsection[task]{\tt character(len=20) :: task}

Tells the code what to compute. 

The valid options for this parameter are:
\begin{itemize}
\item[{\bf --}]  \verb#dos# (default)
\item[{\bf --}]  \verb#compare_dos#
\item[{\bf --}]  \verb#compare_jdos#
\item[{\bf --}]  \verb#jdos#
\item[{\bf --}]  \verb#pdos#
\item[{\bf --}]  \verb#optics#
\item[{\bf --}]  \verb#core#
\item[{\bf --}]  \verb#all#
\end{itemize}
Several tasks can be specified \emph{e.g.} to compute dos and jdos use
\verb#task : dos jdos#.  
However, the \verb#compare_dos# and \verb#compare_jdos# tasks can only be combined with each other, and no additional tasks.
\verb#compare_dos# and \verb#compare_jdos# calculate the DOS and JDOS respectively using all broadening schemes. It is good practice to check the quality of the underlying DOS before other tasks are requested. 

\subsection[broadening]{\tt character(len=50) :: broadening}

Specified the scheme used to broaden a discrete sampling of the
Brillouin Zone to a continuous spectral function.

The valid options for this parameter are:
\begin{itemize}
\item[{\bf --}]  \verb#adaptive# (default)
\item[{\bf --}]  \verb#fixed#
\item[{\bf --}]  \verb#linear#
\item[{\bf --}]  \verb#quad# (not currently implemented)
\end{itemize}

\subsection[iprint]{\tt integer :: iprint}

This indicates the level of verbosity of the output from 1,
the bare minimum: 2, with progress reports: to 3, which corresponds to full debugging output.

The default value is 1.


\subsection[energy\_unit]{\tt character(len=20) :: energy\_unit}
The energy unit to be used for writing quantities in the output files.

The valid options for this parameter are:
\begin{itemize}
\item[{\bf --}]  \verb#eV# (default)
\item[{\bf --}]  \verb#Ry#
\item[{\bf --}]  \verb#Ha#
\end{itemize}

\subsection[adaptive\_smearing]{\tt logical :: legacy\_file\_format}
\label{sect:lff}
\begin{itemize}
\item[{\bf --}] \verb#TRUE#  Read \castep\ input compatible with versions $<$ 6.0.
\item[{\bf --}] \verb#FALSE# (Default) Read \castep\ input compatible for use with \castep\ versions 6.0$+$ and generated with the castep spectral task.
\end{itemize}

\subsection[adaptive\_smearing]{\tt real(kind=dp) :: adaptive\_smearing}
Set the relative smearing in the adaptive scheme.

Default value is 0.4

\subsection[fixed\_smearing]{\tt real(kind=dp) :: fixed\_smearing}
Smearing width for fixed broadening.

If $\verb#spectral_scheme# = \verb#fixed#$ default value is 0.3eV.


\subsection[adaptive\_smearing]{\tt real(kind=dp) :: linear\_smearing}
Smear the linear broadening with a Gaussian of this width. 

Default value is 0.0.

\subsection[compute\_efermi]{{\tt character(len=20) :: efermi}}
Choose which Fermi energy to use.

The valid options for this parameter are:
\begin{itemize}
\item[{\bf --}]  \verb#optados# (default) \optados\ recalculates the Fermi energy by performing a DOS calculation. 
\item[{\bf --}]  \verb#file# Take the value from the output of the ab-initio calculation.
\item[{\bf --}]  \verb#insulator# Assume that the material is an insulator and counts filled bands to find the Fermi energy.
\item[{\bf --}]  \verb#<real number># User supplied value.
\end{itemize}

The default value is {\tt optados}.


\subsection[output\_format]{\tt character(len=20) :: output\_format}
Format in which to output data.

The valid options for this parameter are:
\begin{itemize}
\item[{\bf --}]  \verb#gnuplot# 
\item[{\bf --}]  \verb#grace# (default)
\end{itemize}

\subsection[finite\_bin\_correction]{\tt logical :: finite\_bin\_correction}
Force each Gaussian to be larger than a single energy bin. (Useful for adaptive smearing and semi-core states when \verb#numerical_intdos=TRUE#). 

Default value \verb#TRUE#.

\subsection[numerical\_intdos]{\tt logical :: numerical\_intdos}
Calculate the integrated dos by numerical integration instead of semi-analytically. (Useful for comparison with \lindos.)

Default value \verb#FALSE#.

\subsection[hybrid\_linear]{\tt logical :: hybrid\_linear}
Switch from linear broadening scheme to adaptive broadening when band gradient less than \verb#hybrid_linear_grad_tol#. 
This allows for a good description of very flat bands such as defect and semi-core states. May also be used in conjunction 
with \verb#finite_bin_correction# further improving the DOS and band energy

Default value \verb#FALSE#.

\subsection[hybrid\_linear\_grad\_tol]{\tt real(kind=dp) :: hybrid\_linear\_grad\_tol}
Tolerance for switching from linear to adaptive broadening when using hybrid\_linear option.

The default value is 0.01eV/\AA.


\subsection[devel\_flag]{\tt character(len=50) :: devel\_flag}

Not a regular keyword. Its purpose is to allow a developer to pass a
string into the code to be used inside a new routine as it is developed.

No default.



\section{DOS Parameters}

\subsection[compute\_band\_energy]{\tt logical :: compute\_band\_energy}
Compute the band energy by summing bands both using \castep's eigenvalue and \optados's density of states.

Default value \verb#TRUE#.

\subsection[compute\_band\_energy]{\tt logical :: compute\_band\_gap}
Compute the optical, thermal and average band gap.

Default value \verb#FALSE#.

\subsection[DOS\_per\_volume]{\tt logical :: dos\_per\_volume}
Present DOS per simulation cell volume.

Default value \verb#FALSE#.

\subsection[dos\_min\_energy]{\tt real(kind=dp) :: dos\_min\_energy}
Lower energy range for DOS and related properties.

Default value is 5eV below the lowest eigenvalue in the bands file.

\subsection[dos\_max\_energy]{\tt real(kind=dp) :: dos\_max\_energy}
Upper energy range for DOS and related properties.

Default value is 5eV above the highest eigenvalue in the bands file.


\subsection[dos\_nbins]{\tt real(kind=dp) :: dos\_nbins}
Instead of setting a Default value \verb#dos_spacing# the total number of DOS bins may be given.  (Useful for comparison with \lindos.)

\subsection[dos\_spacing]{\tt real(kind=dp) :: dos\_spacing}
Resolution at which to compute the DOS and related properties.

Default value is 0.1eV 

\subsection[set\_efermi\_zero]{\tt logical :: set\_efermi\_zero}
Shift energy scales so that the Fermi energy is at 0.

Default value \verb#FALSE#.



\section{JDOS Parameters}

\subsection[jdos\_max\_energy]{\tt real(kind=dp) :: jdos\_max\_energy}
Upper energy range for JDOS and related properties.

Default value is the difference between the valence band maximum (or
Fermi level) and the highest eigenvalue in the bands file.

\subsection[jdos\_spacing]{\tt real(kind=dp) :: jdos\_spacing} 
Resolution at which to compute the DOS and related properties.

Default value is 0.01eV. 

\subsection[scissor\_op]{\tt real(kind=dp) :: scissor\_op}
Value of the scissor operator. 

Default value is 0\,eV (\emph{i.e.} not used)

\subsection[exclude\_bands]{\tt real(kind=dp) :: exclude\_bands}
This allows a list of bands which are NOT to be included in the JDOS or OPTICS calculation to be specified.  The bands 1, 2, 4, 5, 6, for example, can be specified using \verb#exclude_bands = 1,2,4-6#. 

\section{PDOS Parameters}

\subsection[pdos]{\tt character :: pdos}
Defines which components to include in the pdos analysis:

\begin{itemize}
\item[{\bf --}]  \verb#angular# (decompose as s,p,d \emph{etc.})
\item[{\bf --}]  \verb#species#    (decompose onto atomic species C, H \emph{etc.})
\item[{\bf --}]  \verb#sites#    (decompose onto atomic sites C1, H1, H2 \emph{etc.})
\item[{\bf --}]  \verb#species_ang#    (decompose onto angular momentum channels and species Cs, Cp \emph{etc.})
\item[{\bf --}]  \verb#C:H#     (decompose onto Carbon and Hydrogen sites)
\item[{\bf --}]  \verb#C1:C3:C4-C8#  (decompose onto atoms C1, C2 and C4,C5,C6,C7,C8)
\item[{\bf --}]  \verb#Si1(s;d)#     (decompose onto 's' and 'd' channels for
  atom Si1)
\item[{\bf --}]  \verb#sum:C1:C3:C4-C8#  (decompose onto atoms C1, C2 and C4,C5,C6,C7,C8 and combine into the single projection)

\end{itemize}

\section{Optics Parameters}

\subsection[optics\_geom]{\tt character(len=20) :: optics\_geom}

Specifies the geometry for the optics calculation.  Possible options:
\begin{itemize}
\item[{\bf --}]  \verb#polycrystalline# (Isotropic average)
\item[{\bf --}]  \verb#polarized#  
\item[{\bf --}]  \verb#unpolarized# 
\item[{\bf --}]  \verb#tensor# (Full dielectric tensor)
\end{itemize}
The default is polycrystalline.

\subsection[optics\_qdir]{\tt real(kind=dp) :: optics\_qdir(3)}
Direction of polarisation. Must be specified if \verb#optics_geom = polarized#  
or \verb#optics_geom = unpolarized#.

There is no default value

\subsection[optics\_intraband]{\tt logical :: optics\_intraband}
If true, the intraband contribution to the dielectric function will be calculated.  (Important for metals.)    

The default is FALSE.

\subsection[optics\_drude\_broadening]{\tt real(kind=dp) :: optics\_drude\_broadening}
Value of broadening included in the Drude term expressed in $s^{-1}$.  

The default value is 1E-14.  

\subsection[optics\_lossfn\_broadening]{\tt real(kind=dp) :: optics\_lossfn\_broadening}
FWHM of Gaussian used to broaden the loss function.  

The default value is 0 ($i.e.$ no broadening is used).  


\section{Core-level Parameters}

\subsection[core\_geom]{\tt character(len=20) :: core\_type}

Determines if we want absorption (transition from core to conduction
ELNES / XANES)
or emission (transition from valence to core XAS). It is also possible
to plot both.
\begin{itemize}
\item[{\bf --}]  \verb#absorption# (default)
\item[{\bf --}]  \verb#emission#
\item[{\bf --}]  \verb#all#
\end{itemize}

\subsection[core\_geom]{\tt character(len=20) :: core\_geom}

Specifies the geometry for the core-spectra calculation.  Possible options: 
\begin{itemize}
\item[{\bf --}]  \verb#polycrystalline# (Isotropic average)
\item[{\bf --}]  \verb#polarized#  
\end{itemize}
The default is polycrystalline.

\subsection[core\_qdir]{\tt real(kind=dp) :: core\_qdir(3)}
Direction of polarisation. Must be specified if \verb#core_geom = polarized#.

There is no default value.

\subsection[core\_LAI\_broadening]{\tt logical :: core\_LAI\_broadening}
Include life-time and instrumentation broadening.  

The default is FALSE.

\subsection[core\_gaussian\_width]{\tt real(kind=dp) :: LAI\_gaussian\_width}
FWHM of Gaussian function used to broaden spectrum.  

The default value, if \verb#core_LAI_broadening = true#, is 0 ($i.e.$ no Gaussian used).

\subsection[core\_lorentzian\_width]{\tt real(kind=dp) :: LAI\_lorentzian\_width}
FWHM of fixed Lorentzian function used to broaden spectrum.  

The default value, if \verb#core_LAI_broadening = true#, is 0 ($i.e.$ no fixed Lorentzian used).

\subsection[core\_lorentzian\_scale]{\tt real(kind=dp) :: LAI\_lorentzian\_scale}
Variation of Lorentzian function with energy i.e. the width of the Lorentzian is (energy above the Fermi level x 
\verb#core_lorentzian_scale#).  If set to zero, no energy dependent broadening is included.  
If \verb#core_lorentzian_scale# and \verb#core_lorentzian_width# are both specified, the 
total width of the Lorentzian used will be \verb#core_lorentzian_width# + (energy above the Fermi level x 
\verb#core_lorentzian_scale#).

The default value, if \verb#core_LAI_broadening = true#, is 0.1. 

\subsection[core\_lorentzian\_offset]{\tt real(kind=dp) :: LAI\_lorentzian\_offset}
Energy (in eV) above the Fermi level that the energy dependent broadening starts.  

The default value, if \verb#core_LAI_broadening = true#, is 0.


\chapter{Examples}

Each example in the \verb#examples/# directory contains example \castep\ input files and a sample \optados\ input file. To keep the \optados\ distribution light \castep\ output files for the examples have not been provided. You need to run \castep\ on the \castep\ input files before running \optados\ on the examples.

\section{Density of States}
\begin{itemize}
\item \emph{Outline} : This is a simple example of using \optados\ for calculating electronic density of states of crystalline silicon in a 2 atom cell.
\item \emph{Input Files}:
\begin{itemize}
\item \verb#examples/Si2_DOS/Si2_dos.cell# - The \castep\ cell file containing information about the simulation cell.
\item \verb#examples/Si2_DOS/Si2_dos.param# - The \castep\ param file containing information about the parameters for the SCF and spectral calculations.
\item \verb#examples/Si2_DOS/Si2_dos.odi# - The \optados\ input file, containing the parameters necessary to run \optados.
\end{itemize}
\end{itemize}

\begin{enumerate}
%%%%  Step 1 %%%%%
\item Perform a \castep\ calculation on the bulk silicon using the  \verb#Si2_dos.cell#  and \verb#Si2_dos.param# input files. 

  \verb#$ castep Si2# 

This should take a couple of seconds to run. More help can be found in  the tutorials  on  the \castep\ website \verb#www.castep.org#.  

%%%%  Step 2 %%%%%
\item Perform an \optados\ calculation.  Add \verb#LEGACY_FILE_FORMAT : true# in the \verb#Si2_DOS.odi# input file, if the \castep\ version you are using is before 6.0. Then execute:   
 
\verb#$ optados.<SYSTEM>.<BUILD>.<COMMS_ARCH>.x86_64 Si2# 

This generates 3 files:
\begin{itemize}
\item \verb#Si2.odo# -- \optados\ general output file.
\item \verb#Si2.adaptive.dat# -- The adaptive broadened DOS raw output data.
\item  \verb#Si2.adaptive.agr# -- The adaptive broadened DOS in a file suitable to be plotted by  \verb#xmgrace#.   
\end{itemize}

%%%%  Step 3 %%%%%
\item Open the \verb#Si2.odo# file in a text editor (\emph{e.g.} vi or emacs).  \optados\ has performed a Density of States calculation.

\begin{verbatim}
+------------------------ Fermi Energy Analysis ------------------------------+
| From Adaptive broadening                                                    |
|   Spin Component : 1 occupation between    3.99961 and    4.00003    <- Oc  |
|   Spin Component : 2 occupation between    3.99961 and    4.00003    <- Oc  | 
|         Fermi energy (Adaptive broadening) :   5.4109 eV             <- EfA |
+-----------------------------------------------------------------------------+
\end{verbatim}
It has used the integrated DOS to work out the Fermi level, and has suggested the error in the integration by indicating the number of electrons at the Fermi level. Since we had 4 up electrons and 4 down in the input file this analysis seems satisfactory.
\begin{verbatim}
+----------------------- Electronic Data ------------------------------------+
|  Number of Bands                           :            23                 |
|  Grid size                                 :       10 x 10 x 10            |
|  Number of K-points                        :           110                 |
|  Spin-Polarised Calculation                :           True                |
|  Number of up-spin electrons               :             4.00              |
|  Number of down-spin electrons             :             4.00              |
+----------------------------------------------------------------------------+
\end{verbatim}

Since we had \verb#efermi : optados#, \optados\ sets the internal value of the Fermi level to the one it has derived from the DOS. This is important for subsequent calculations. Other valid options are \verb#file#, where \optados\ uses the value calculated by the electronic structure code that generated the eigenvalues;  \verb#insulator#, where \optados\ uses a value calculated from assuming the system is non-metallic; or a value set by the user.

\optados\ now performs some analysis of the DOS at the Fermi level,
\begin{verbatim}
+----------------------- DOS at Fermi Energy Analysis ------------------------+
|                          Fermi energy used :   5.4109 eV                    |
| From Adaptive broadening                                                    |
|   Spin Component : 1   DOS at Fermi Energy :   0.0011 eln/cell       <- DEA |
|   Spin Component : 2   DOS at Fermi Energy :   0.0011 eln/cell       <- DEA |
+-----------------------------------------------------------------------------+
\end{verbatim}
From this we may assume that there is a band gap.

Importantly, then \optados\ calculates the band energy from the DOS is has calculated.
\begin{verbatim}
+--------------------------- Band Energy Analysis ----------------------------+
|          Band energy (Adaptive broadening) :       1.3609 eV         <- BEA |
|                  Band energy (From CASTEP) :       1.3622 eV         <- BEC |
+-----------------------------------------------------------------------------+
\end{verbatim}
As the quality of the \optados\ calculation is increased these two values should converge to the same answer.

Finally \optados\ shifts the Fermi level to 0\,eV, for the output files.

%%%%  Step 4 %%%%%
\item  The DOS is outputted to {\tt Si2.adaptive.dat}. This contains 5 columns as described in the header of the file:
\begin{verbatim}
########################################################################
#
#                  O p t a D O S   o u t p u t   f i l e 
#
#    Density of States using adaptive broadening
#  Generated on 12 Feb 2012 at 16:50:37 
# Column        Data
#    1        Energy (eV)
#    2        Up-spin DOS (electrons per eV)
#    3        Down-spin DOS (electrons per eV)
#    4        Up-spin Integrated DOS (electrons)
#    5        Down-spin Integrated DOS (electrons)
#
########################################################################
\end{verbatim}

This file can be plotted by your favourite graph-plotting software. However, \optados\ has made things easy and generated a  {\tt Si2.adaptive.agr} file which is directly plottable using \verb#xmgrace# as shown in Fig.\,\ref{fig:Si2_adaptive1}.

\verb#$ xmgrace Si2.adaptive.agr#.

\begin{figure}[h]
\begin{center}
\scalebox{0.35}{\includegraphics{./images/Si2.adaptive.eps}}
\caption{Density of States of Silicon generated by adaptive broadening and a very coarse energy sampling of 0.1 eV.}
\label{fig:Si2_adaptive1} 
\end{center}
\end{figure}

%%%%  Step 5 %%%%%
\item We now try again with a better sampling of the DOS, by setting \verb#DOS_SPACING : 0.001# and also analyse the band gap, by setting \verb#COMPUTE_BAND_GAP : true#.  You can remove all of the \optados\ output files by using \verb#./tools/optados_clean# in your working directory.  If you have a parallel version of \optados\ compiled, now might be the time to try it out, if not, the serial version will be fine, but just take a bit longer.  You can set \verb#IPRINT : 2# to see a progress report in  \verb#Si2.odo#.  In parallel:

\verb#$ mpirun -np <nprocs> optados.SYSTEM.BUILD.COMMS_ARCH.x86_64 Si2# 

but your MPI implementation may be different.

\item In  \verb#Si2.odo# we now have a new section analysing the band gap in various ways.
\begin{verbatim}
+----------------------------- Bandgap Analysis ------------------------------+
|          Number of kpoints at       VBM       CBM                           |
|                   Spin :   1  :      1         1                            |
|                   Spin :   2  :      1         1                            |
|               Thermal Bandgap :   0.6676272107  eV                   <- TBg |
|            Between VBM kpoint :    0.05000    0.05000    0.05000            |
|                 and CBM kpoint:   -0.45000   -0.05000   -0.45000            |
|             ==> Indirect Gap                                                |
+-----------------------------------------------------------------------------+
|                            Optical Bandgap                                  |
|                   Spin :   1  :     2.5542517447  eV                 <- OBg |
|                   Spin :   2  :     2.5542463024  eV                 <- OBg |
|          Number of kpoints with this gap                                    |
|                   Spin :   1  :           1                                 |
|                   Spin :   2  :           1                                 |
+-----------------------------------------------------------------------------+
|                            Average Bandgap                                  |
|                   Spin :   1  :     3.8121372691  eV                 <- ABg |
|                   Spin :   2  :     3.8121342659  eV                 <- ABg |
|              Weighted Average :     3.8121357675  eV                 <- wAB |
+-----------------------------------------------------------------------------+
\end{verbatim}

\optados\ is very careful in its band gap analysis. It uses the bare eigenvalues (un-broadened) and works out the nature and size of the thermal gap, optical gap and the average gap over all of the Brillouin zone. In cases of multi-valleyed semiconductors \optados\ will report the number of conduction band minima or valence band maxima with identical energies, but will not report the nature of the gap. 

Increasing the number of integration points has improved the band energy of the adaptive smearing:
\begin{verbatim}
|          Band energy (Adaptive broadening) :       1.3623 eV         <- BEA |
\end{verbatim}


%%%%  Step 6 %%%%%
\item  Now set {\tt TASK : compare\_dos} and re-run \optados. \optados\ will calculate DOS using all the broadening methods, this is good practice to see whether the broadening widths are appropriate before more advanced tasks are carried out, such as Joint-DOS, core and optical calculations.

Plotting the linear broadened DOS over the adaptive we see that the default adaptive broadening is appropriate,

\verb#xmgrace Si2.adaptive.agr -nxy Si2.linear.dat#

Linear broadening, although a massive improvement over fixed broadening, sometimes appears noisy if used with very low numbers of k-points. Linear and adaptive DOS should be compared and \verb#ADAPTIVE_SMEARING# may be tuned by eye until the adaptive DOS contains the features of the linear DOS, but with less noise.  Adding a random shift to a k-point mesh can greatly increase the quality of the DOS, especially if the k-point set contained the $\Gamma$-point. It generally pays off in computational time to have a coarse mesh at low symmetry points, than a fine mesh centred on high symmetry points. 

Both fixed and adaptive broadening can fail to plot the sharpest features, such as semi-core states, if the bin widths are too broad. These sharp features may be forced to be present using the narrowest Gaussian still reproducible by the chosen bin widths by setting \verb#FINITE_BIN_CORRECTION : true#.  

Linear broadening may also be improved by using \verb#HYBRID_LINEAR : true#. Van Hove singularities and other sharp features are now described at the adaptive broadening level if the bands are flatter than \verb#HYBRID_LINEAR_GRAD_TOL#. Hybrid linear may also take advantage of the finite bin correction if required.

\item Compare the fixed and adaptive DOS and see the advantage of adaptive broadening over standard Gaussian smearing.

\end{enumerate}


\section{Projected Density of States}
We assume the reader is familiar with the previous section on Density of States calculations and is now familiar with choosing broadening widths, and running \optados. 
\begin{itemize}
\item \emph{Outline} : This is a simple example of using \optados\ for calculating electronic density of states of 2 atoms of crystalline silicon projected onto LCAO basis states.
\item \emph{Input Files}:
\begin{itemize}
\item \verb#examples/Si2_PDOS/Si2_dos.cell# - The \castep\ cell file containing information about the simulation cell.
\item \verb#examples/Si2_PDOS/Si2_dos.param# - The \castep\ param file containing information about the parameters for the SCF and spectral calculations.
\item \verb#examples/Si2_PDOS/Si2_dos.odi# - The \optados\ input file, containing the parameters necessary to run \optados.
\end{itemize}
\end{itemize}

\begin{enumerate}

\item Choose a broadening scheme for the projected-DOS calculation and test using {\tt TASK : compare\_dos} as explained in the previous example. Checking that the  {\tt DOS\_SPACING} is sufficiently fine for the band energies to match.

\item Once the DOS looks suitable, switch to  {\tt TASK : pdos}. We choose to decompose the DOS into angular momentum channels ({\tt PDOS : angular}) and as in the previous example we choose to recalculate the Fermi level using the calculated DOS, rather than use the Fermi level suggested by \castep. 

\item Execute \optados.

\item  The output can be found in {\tt Si2.pdos.dat}.
\begin{verbatim}
##############################################################################
#
#                  O p t a D O S   o u t p u t   f i l e
#
#  Generated on 13 Feb 2012 at 10:15:10
##############################################################################
#+----------------------------------------------------------------------------+
#|                    Partial Density of States -- Projectors                 |
#+----------------------------------------------------------------------------+
#| Projector:    1 contains:                                                  |
#|           Atom            AngM Channel                                     |
#|          Si   1                 s                                          |
#|          Si   2                 s                                          |
#+----------------------------------------------------------------------------+
#| Projector:    2 contains:                                                  |
#|           Atom            AngM Channel                                     |
#|          Si   1                 p                                          |
#|          Si   2                 p                                          |
#+----------------------------------------------------------------------------+
#| Projector:    3 contains:                                                  |
#|           Atom            AngM Channel                                     |
#|          Si   1                 d                                          |
#|          Si   2                 d                                          |
#+----------------------------------------------------------------------------+
#| Projector:    4 contains:                                                  |
#|           Atom            AngM Channel                                     |
#|          Si   1                 f                                          |
#|          Si   2                 f                                          |
#+----------------------------------------------------------------------------+
\end{verbatim}
The header shows that there are four projectors described below. The first containing the s-channels of both silicon atoms, the second the p-channels \emph{etc.}

\item The output is easily plotted using xmgrace:

\verb#xmgrace -nxy Si2.pdos.dat#

\item Setting {\tt DOS\_SPACING : 0.001} gives a high quality plot, as shown in Fig\,\ref{fig:Si2_PDOS}

\begin{figure}[h]
\begin{center}
\scalebox{0.35}{\includegraphics*{./images/Si2_PDOS.eps}}
\caption{Density of States of Silicon generated by adaptive broadening projected onto LCAO momentum states.}
\label{fig:Si2_PDOS} 
\end{center}
\end{figure}


\item Other things to try are:
\begin{itemize}
\item[{\bf --}]  \verb#PDOS : Si1;Si2(s)#  -- Output the PDOS on Si atom 1 and the PDOS on the s-channel of Si atom 2. (Resulting in two projectors)
\item[{\bf --}]  \verb#PDOS : sum:Si1-2(s)#  --  Output the sum of the s-channels on the two Si atoms. (Resulting in one projector)
\item[{\bf --}]  \verb#PDOS : Si1(p)# -- Output the p-channel on Si atom 1. (Resulting in one projector)
\end{itemize}
\end{enumerate}


\section{JDOS}
See  \verb#examples/Si2_JDOS/#. This is a simple example of using \optados\ for calculating joint electronic density of states.  We choose to recalculate the Fermi level using the calculated DOS, rather than use the Fermi level suggested by \castep\, and so \verb#EFERMI: OPTADOS# is included in the \verb#Si2.odi# file.  
\begin{enumerate}
\item Execute \optados\ using the example files.  The JDOS is outputted to {\tt Si2.jadaptive.dat}. A file suitable for plotting using {\tt xmgrace} is written to {\tt Si2.jadaptive.agr}.
\item Check the effect of changing the sampling by increasing and decreasing the value of \verb#JDOS_SPACING# in the \verb#Si2.odi# file. 
\item If {\tt TASK : compare\_jdos} is used instead, \optados\ will calculate the JDOS using all the broadening methods.  This is good practice to see whether the broadening widths are appropriate before more advanced tasks are carried out.  
\end{enumerate}

\section{OPTICS}
Two sets of example files are provided for calculations of optical properties.  For each example, the \castep\ files containing all the cell and simulation parameters are included, along with an \optados\ input file. We assume that the reader is familiar with the previous sections on DOS and JDOS.  
\begin{itemize}
\item[{\bf --}]  \verb#examples/Si2_OPTICS/#  This is a simple example of using \optados\ to calculate the optical properties of crystalline silicon, which is an insulator.   
\item[{\bf --}]\verb#examples/Al_OPTICS/#  This is a simple example of using \optados\ to calculate the optical properties of a metal, aluminium. 
\end{itemize}

\subsection{Silicon}
\begin{enumerate}
\item Choose a broadening scheme for the JDOS as explained in the JDOS example.  Once the JDOS looks suitable, switch the \verb#task# from \verb#JDOS# to \verb#OPTICS# and execute \optados\ to calculate the optical properties.  Several \verb#*.dat# files are produced:
\begin{itemize}
\item[{\bf --}] \verb#Si2_OPTICS_absorption.dat# : This file contains the absorption coefficient (second column) as function of energy (first column).
\item[{\bf --}] \verb#Si2_OPTICS_conductivity.dat# : This file contains the conductivity outputted in SI units (Siemens per metre).  The columns are the energy, real part  and imaginary part of the conductivity respectively.  
\item[{\bf --}] \verb#Si2_OPTICS_epsilon.dat# : This file contains the dielectric function.  The columns are the energy and real and imaginary parts of the dielectric function respectively. The file header also includes the result of the sum rule $\int_o^{\omega'}Im\epsilon{}(\omega{})d\omega = N_{eff}(\omega')$.  $N_{eff}$ is the effective number of electrons contributing to the absorption process, and is a function of energy.    
\item[{\bf --}] \verb#Si2_OPTICS_loss_fn.dat# : This file contains the loss function (second column) as a function of energy (first column).  The header of the file shows the results of the two sum rules associated with the loss function $\int_o^{\omega'}Im\Big(\frac{-1}{\epsilon{}(\omega{})}\Big)\,\omega{}\,d\omega = N_{eff}(\omega')$ and $\int_o^{\omega'}Im\Big(\frac{-1}{\epsilon{}(\omega{})}\Big)\,\frac{1}{\omega}\,d\omega = \frac{\pi}{2}$.
\item[{\bf --}] \verb#Si2_OPTICS_reflection.dat# : This file contains the reflection coefficient (second column) as a function of energy (first column).
\item[{\bf --}] \verb#Si2_OPTICS_refractive_index.dat# : This file contains the refractive index.  The columns are the energy and real and imaginary parts of the refractive index respectively.  
\end{itemize}
Corresponding \verb#*.agr# files are also generated which can be plotted easily using xmgrace.

\item Change parameters \verb#JDOS_SPACING# and \verb#JDOS_MAX# and check the effect on the optical properties.  Note: all of the other optical properties are derived from the dielectric function.  

\item  The \optados\ input file has been set up to calculate the optical properties in the polycrystalline geometry (\verb#optics_geom = polycrystalline#).  It is possible to calculate either polarised or unpolarised geometries, or to calculate the full dielectric tensor.  To calculate the full dielectric tensor set \verb#optics_geom = tensor#.  This time only the file \verb#Si2_OPTICS_epsilon.dat# is generated.  The format of this file is the same as before (the columns are the energy and the real and imaginary parts of the dielectric function respectively), but this time the six different components of the tensor are listed sequentially in the order $\epsilon_{xx}$, $\epsilon_{yy}$, $\epsilon_{zz}$, $\epsilon_{xy}$, $\epsilon_{xz}$ and $\epsilon_{yz}$.

\item Additional broadening can be included in the calculation of the loss function.  This is done by including the keyword \verb#optics_lossfn_broadening# in the \optados\ input file.  If you include this keyword and re-run \optados, you will find that the file \verb#Si2_OPTICS_loss_fn.dat# now has three columns.  These are the energy, unbroadened spectrum and broadened spectrum respectively.  

\end{enumerate}

\subsection{Aluminium}
\begin{enumerate}
\item As for the silicon example, a broadening scheme for the JDOS should first be determined.  
\item Aluminium is a metal so we need to include both the interband and intraband contributions to the dielectric function.  To include the intraband contribution \verb#optics_intraband# \verb#= true# must be included in the \optados\ input file.  When you run \optados\, the same files are generated as when only the interband term is included.  

The \verb#Al_OPTICS_epsilon.dat# file has the same format as before, but it now contains sequentially the interband contribution, the intraband contribution and the total dielectric function.  (The file \verb#Al_OPTICS_epsilon.agr# only contains the interband term.)  In the same way, \verb#Al_OPTICS_loss_fn.dat# contains the interband contribution, intraband contribution and total loss function.  All other optical properties are calculated from the total dielectric function and the format of the output files remains the same.

\item In the case where the dielectric tensor is calculated and the intraband term is included, only the \verb#Al_OPTICS_epsilon.dat#  file is generated.  As before it contains each component, but this time it lists sequentially the interband contribution, intraband contribution and total dielectric function for each component.   

\item This time, if additional broadening for the loss function is included by using the key word \verb#optics_lossfn_broadening#, \verb#AL_OPTICS_loss_fn.dat# will contains four sequential data sets.  These are the interband contribution, the intraband contribution, the total loss function without the additional broadening and the broadened total loss function.  
\end{enumerate}

\section{CORE}
See  \verb#examples/Si2_CORE/#. This is a simple example of using \optados\ for calculating core level absorption spectra for crystalline silicon.  We assume that the reader is familiar with the previous section on calculating DOS.  
\begin{enumerate}
\item We begin by running a \castep\ calculation using the files provided in \verb#examples/Si2_CORE#.  Note that we do not specify pseudopotentials in the \verb#Si2_CORE.cell# file hence require \castep\ to generate on-the-fly pseudopotentials.  This is important as most pseudopotential formats do not contain enough information for the PAW reconstruction needed for a CORE calculation. For any atom a CORE spectra is to be calculated an On-the-fly pseudopotential must be used.

Execute \optados\ using the \optados\ input file provided and the file \verb#Si2_CORE_core_edge.dat# will be created.  The file contains two columns, the first is the energy and the second is the spectrum.  This file contains the following edges:

\begin{verbatim}
 # Si 1    K1
 # Si 1    L1
 # Si 1  L2,3
 # Si 2    K1
 # Si 2    L1
 # Si 2  L2,3
\end{verbatim}
$i.e.$ all edges from all atoms are produced.    

\item To include a core-hole in the calculation, first one atom is chosen to have the excitation.  To begin, we will keep two atoms in the unit cell and distinguish one atom by changing the \verb#%BLOCK POSITIONS_FRAC#
\begin{verbatim} 
     %BLOCK POSITIONS_FRAC
            Si:exi    0.0000000000    0.0000000000    0.0000000000
            Si        0.2500000000    0.2500000000    0.2500000000
      %ENDBLOCK POSITIONS_FRAC
\end{verbatim}
in the \verb#Si2_CORE.cell# file.  The atom \verb#Si:exi# is the one to have a core-hole.  To create a core-hole we remove a 1s electrons from the electronic configuration used in the generation of the pseudopotential.  We have already generated an on-the-fly pseudopotential without a core-hole in the previous section.  Information about the pseudopotentials is included at the top of the \verb#Si2_CORE.castep# file.  

\begin{verbatim}
   ============================================================                
   | Pseudopotential Report - Date of generation 16-05-2012   |                
   ------------------------------------------------------------                
   | Element: Si Ionic charge:  4.00 Level of theory: LDA     |                
   |                                                          |                
   |               Reference Electronic Structure             |                
   |         Orbital         Occupation         Energy        |                
   |            3s              2.000           -0.400        |                
   |            3p              2.000           -0.153        |                
   |                                                          |                
   |                 Pseudopotential Definition               |                
   |        Beta     l      e      Rc     scheme   norm       |                
   |          1      0   -0.400   1.797     qc      0         |                
   |          2      0    0.250   1.797     qc      0         |                
   |          3      1   -0.153   1.797     qc      0         |                
   |          4      1    0.250   1.797     qc      0         |                
   |         loc     2    0.000   1.797     pn      0         |                
   |                                                          |                
   | Augmentation charge Rinner = 1.298                       |                
   | Partial core correction Rc = 1.298                       |                
   ------------------------------------------------------------                
   | "2|1.8|1.8|1.3|2|3|4|30:31:32LGG(qc=4)"                  |                
   ------------------------------------------------------------                
   |      Author: Chris J. Pickard, Cambridge University      |                
   ============================================================      
\end{verbatim}

The line 
\begin{verbatim}
    2|1.8|1.8|1.3|2|3|4|30:31:32LGG(qc=4)
\end{verbatim}
specifies the parameters used to create the pseudopotential.  We use this as the starting point and then remove one of the core 1s electrons to create a core-hole pseudopotential.  This is done by including \verb#{1s1.00}# in the pseudopotential string as shown:
\begin{verbatim}
    2|1.8|1.8|1.3|2|3|4|30:31:32LGG{1s1.00}(qc=4)
\end{verbatim}
If, instead of removing a 1s electron, we wanted to remove a 2s electron from the core, we would have included \verb#{2s1.00}# instead of \verb#{1s1.00}# in the pseudopotential string.

We are only interested in the spectra from the atom with the core-hole and so copy the pseudopotential file generated by the previous calculation (\verb#Si_OFT.usp#) to \verb#Si_LDA.usp#.  Then include 
\begin{verbatim}
      %BLOCK SPECIES_POT
            Si:exi    2|1.8|1.8|1.3|2|3|4|30:31:32LGG{1s1.00}(qc=4)
            Si        Si_LDA.usp
      %ENDBLOCK SPECIES_POT
\end{verbatim}
in the \castep\ \verb#Si2_CORE.cell# file.  


To maintain the neutrality of the cell, we include 
\begin{verbatim}
  CHARGE : +1 
\end{verbatim}
in the \verb#Si2_CORE.param# file.  Run the calculation.  This time the \verb#Si2_CORE_core_edge.dat# file will contain only the edges from the core-hole atom.  Compare the K-edge from the core-hole calculation with the previous non-core-hole calculation.  

\item The periodic images of the core-hole will interact with one another.  As this is unphysical, we need to increase the distance between the core-holes.  This is done by creating a supercell.  To start with we use a face-centred unit cell rather than the primitive unit cell.  This is done by changing the lattice parameters and fractional co-ordinates to:
\begin{verbatim} 
     %BLOCK LATTICE_CART 
            5.46  0.00 0.00
            0.00  5.46 0.00
            0.00  0.00 5.46
     %ENDBLOCK  LATTICE_CART

     %BLOCK POSITIONS_FRAC
            Si:exi    0.0000000000    0.0000000000    0.0000000000
            Si        0.5000000000    0.5000000000    0.0000000000  
            Si        0.5000000000    0.0000000000    0.5000000000
            Si        0.0000000000    0.5000000000    0.5000000000
            Si        0.2500000000    0.2500000000    0.2500000000
            Si        0.7500000000    0.2500000000    0.7500000000
            Si        0.2500000000    0.7500000000    0.7500000000
            Si        0.7500000000    0.7500000000    0.2500000000
      %ENDBLOCK POSITIONS_FRAC
\end{verbatim}
Run \optados\ and compare the spectrum from the face-centred unit cell with that from the primitive unit cell.  Continue constructing larger unit cells until the core-hole spectrum stops changing with increasing separation between the periodic images.  

\item Other things to try include
\begin{itemize}
\item[{\bf --}] Changing the geometry from polycrystalline to polarised
\item[{\bf --}] Including life-time and instrumentation broadening
\end{itemize}
\end{enumerate}

\chapter{Frequently Asked Questions}

\section{\optados\ crashes complaining that it can't read the
  seed.bands or seed.cst\_ome file.}
Which version of \castep\ are you using? See Sect.\,\ref{sect:lff}.

\section{The examples do not work and I'm using \castep\ version 6.0}
\castep\ version 6.0 was released during the transition of the spectral module and features. The easiest way to solve this is to obtain a  copy of \castep\ $>6.0$, where \optados\ will work ``out of the box''.  However, to make the examples work with \castep\ 6.0 the \optados\ examples require a few tweaks outlined below.
\begin{itemize}
\item Replace \verb#SPECTRAL_KPOINTS_MP_GRID# in the cell file with \verb#BS_KPOINTS_MP_GRID# and \verb#OPTICS_KPOINTS_MP_GRID#.
\item Remove \verb#SPECTRAL_TASK : DOS# from the param file.
\item Add \verb#DEVEL_FLAG:old_filename# to the odi file.
\item Read Section\,\ref{pdos_castep_6.0} if you are planning to do a partial density of states calculation.
\end{itemize}

\section{I'm getting odd results for my PDOS calculation when using \castep\ version 6.0} \label{pdos_castep_6.0}
The spectral module in \castep\ 6.0 generates the pdos weights commensurate with the optics grid you have chosen. However this gets overwritten by the main routines. Hence to get the right PDOS weights you must set \verb#PDOS_CALCULATE_WEIGHTS : FALSE# in the param file.

\section{I'd like \optados\ to do X as well}
Contact the developers we're always interested in discussing new functionality.

\section{I'd like to help, what can I do?}
Contact the developers, there's always more functionality that we'd
like to add to the code.  

\section{I think I've found a bug: what should I do?}
\begin{itemize}
\item Check and re-check that it is a bug.  
\item Check the output of the electronic structure code.
\item Check that you're using the latest version of \optados.
\item Email the developers the input and output files with
  \verb#iprint : 3# and as much information about the problem as possible.
\end{itemize}



\begin{appendix}

\chapter{Interface with other codes}
Currently \optados\ is interfaced with \castep\ and \textsc{onetep}.
%
This appendix consists of Fortran95 code which defines the input files for \optados\ so as to aid its interfacing with other electronic structure codes.
%
Presented below are the type declaration statements and write statements that may be used in other electronic structure codes or converters to generate the .bands, .ome\_bin, .pos\_bin and .elnes\_bin files used as inputs to \optados.
%
These code snippets are also available in the \texttt{tools/} directory of the \optados\ distribution.

\section{\texttt{.bands} file}
The \texttt{.bands} file is necessary for any calculation using \optados\ it contains the energy eigenvalues at each k-point and data about the electronic properties of the system.
%
The bands file is a formatted file.
\begin{verbatim}
integer,parameter:: dp=selected_real_kind(15,300)  ! Define double precision
integer:: num_kpoints                ! Number of kpoints
integer:: num_spins                  ! Number of spins
integer:: max_eigenv                 ! Number of bands included in matrix elements
integer:: num_electrons(num_spins)   ! Number of electrons of each spin
integer:: num_eigenvalues(num_spins) ! Number of eigenvalues of each spin
real(dp):: efermi(num_spins)         ! Fermi level calculated from electronic 
                                     ! structure code for each spin           
real(dp):: cell_lattice(1:3,1:3)     ! Real-space lattice vectors
real(dp):: kpoint_positions(num_kpoints,1:3) ! K-point position vectors in 
                                             ! fractions of Brillouin Zone
real(dp):: kpoint_weight(num_kpoints)        ! K-point weight
real(dp):: band_energy(max_eigenv, num_spins, num_kpoints) ! Energy eigenvalue 
                                                           ! list
                                          
write (band_unit,'(a,i4)') 'Number of k-points ', num_kpoints
write (band_unit,'(a,i1)') 'Number of spin components ', num_spins
if(num_spins>1) then
 write (band_unit,'(a,2g10.4)'),'Number of electrons ', num_electrons(:)
else
 write (band_unit,'(a,g10.4)'),'Number of electrons ', num_electrons(:)
endif
if(num_spins>1) then
 write (band_unit,'(a,2i4)'),'Number of eigenvalues ', num_eigenvalues(:)
else
 write (band_unit,'(a,i4)'),'Number of eigenvalues ', num_eigenvalues(:)
endif
if(num_spins>1) then
 write (band_unit,'(a,2f12.6)')'Fermi energy (in atomic units) ', efermi(:)
else
 write (band_unit,'(a,f12.6)')'Fermi energies (in atomic units) ', efermi(:)
endif
write(band_unit,'(a)') 'Unit cell vectors'
write(band_unit,'(3f12.6)') cell_lattice(1,:)
write(band_unit,'(3f12.6)') cell_lattice(2,:)
write(band_unit,'(3f12.6)') cell_lattice(3,:)

do ik=1,num_kpoints
 write (band_unit,'(a,i4,4f12.8)') 'K-point ', ik, &
        & kpoint_positions(num_kpoints,1:3), kpoint_weight(num_kpoints)
 do is=1,num_spins
  write(band_unit,'(a,i1)')'Spin component ',is
  do ib=1,num_eigenvalues(is)
   write (band_unit,*) band_energy(ib,is,ik) 
  end do
 end do
end do
\end{verbatim}
%\begin{landscape}
\section{\texttt{.ome\_bin} file}
The \texttt{.ome\_bin} file is required for adaptive and linear broadening, and for optics calculations. 
%
Note that a \texttt{.dome} file is just the diagonal elements of the \texttt{.ome}.
%
The \texttt{.ome\_bin} file is an unformatted file.
\begin{verbatim}
integer,parameter:: dp=selected_real_kind(15,300) ! Define double precision
real(dp):: file_version=1.0_dp     ! File version
character(len=80):: file_header    ! File header comment
integer:: num_kpoints              ! Number of k-points
integer:: num_spins                ! Number of spins
integer:: max_eigenv               ! Number of bands included in matrix elements
integer:: num_eigenvalues(1:num_spins)   ! Number of eigenvalues per spin channel
complex(dp):: optical_mat(max_eigenv,max_eigenv,1:3,1:num_kpoints,num_spins)! OMEs

write(ome_unit) file_version
write(ome_unit) file_header

do ik=1,num_kpoints
 do is=1,num_spins
  write(ome_unit) (((optical_mat(ib,jb,i,ik,is),ib=1,num_eigenvalues(is)), &
       &jb=1,num_eigenvalues(is)),i=1,3)
 end do
end do
\end{verbatim}
%\end{landscape}
%\begin{landscape}
\section{\texttt{.pdos\_bin} file}
The \texttt{.pdos\_bin} file is required for PDOS calculations.
%
The \texttt{.pdos\_bin} file is an unformatted file.
\begin{verbatim}
integer,parameter:: dp=selected_real_kind(15,300) ! Define double precision
integer:: num_kpoints            ! Number of k-points
integer:: num_spins              ! Number of spins
integer:: num_popn_orb           ! Number of LCAO projectors
integer:: max_eigenv             ! Number of bands included in matrix elements
real(dp):: file_version=1.0_dp   ! File version
integer:: species(1:num_popn_orb)! Atomic species associated with each projector
integer:: ion(1:num_popn_orb)    ! Ion associated with each projector
integer:: am_channel(1:num_popn_orb)     ! Angular momentum channel
integer:: num_eigenvalues(1:num_spins)   ! Number of eigenvalues per spin channel
real(dp):: kpoint_positions(1:num_kpoints,1:3) ! k_x, k_y, k_z in fractions of BZ
real(dp):: pdos_weights(1:num_popn_orb,max_eigenv,num_kpoints,num_spins)!Matrix elements
character(len=80):: file_header ! File header comment

write(pdos_file) file_header
write(pdos_file) file_version
write(pdos_file) num_kpoints
write(pdos_file) num_spins
write(pdos_file) num_popn_orb
write(pdos_file) max_eigenv
write(pdos_file) species(1_:num_popn_orb)
write(pdos_file) ion(1:num_popn_orb)
write(pdos_file) am_channel(1:num_popn_orb)
       
do nk=1,num_kpoints
 write(pdos_file) nk, kpoint_positions(nk,:)
 do ns = 1,num_spins
  write(pdos_file) ns
  write(pdos_file) num_eigenvalues(ns)
  do nb = 1,num_eigenvalues(ns)
   write(pdos_file) real(pdos_weights(num_popn_orb,nb,nk,ns))
  end do
 end do
end do
\end{verbatim}
%\end{landscape}


\begin{landscape}
\section{\texttt{.elnes\_bin} file}
The \texttt{.elnes\_bin} file is required for ELNES calculations.
%
The \texttt{.elnes\_bin} file is an unformatted file.
\begin{verbatim}
integer,parameter:: dp=selected_real_kind(15,300)  ! Define double precision
integer:: tot_core_projectors  ! Total number of core states included in matrix elements
integer:: max_eigenvalues      ! Number of bands included in matrix elements
integer:: num_kpoints          ! Number of k-points
integer:: num_spins            ! Number of spins
integer:: num_eigenvalues(1:num_spins)   ! Number of eigenvalues
integer:: species(1:tot_core_projectors) ! Atomic species associated with each projector
integer:: ion(1:tot_core_projectors)     ! Ion associated with each projector
integer:: n(1:tot_core_projectors)       ! Principal quantum number associated with each projector
integer:: lm(1:tot_core_projectors)      ! Angular momentum quantum number associated with each projector
real(dp):: elnes_mat(tot_core_projectors,max_eigenvalues,3,num_kpoints,num_spins) ! matrix elements

write(elnes_file) tot_core_projectors
write(elnes_file) max_eigenvalues
write(elnes_file) num_kpoints
write(elnes_file) num_spins
write(elnes_file) species(1:tot_core_projectors)
write(elnes_file) ion(1:tot_core_projectors)
write(elnes_file) n(1:tot_core_projectors)
write(elnes_file) lm(1:tot_core_projectors)
        
do nk = 1,num_kpoints
 do ns = 1, num_spins
  write(elnes_file) (((elnes_mat(orb,nb,indx,nk,ns),orb=1,&
       &tot_core_projectors),nb=1,num_eigenvalues(ns)),indx=1,3)
 end do
end do
\end{verbatim}

\end{landscape}
\end{appendix}

\bibliography{bib}

\end{document}

% LocalWords:  odi jdos dos
