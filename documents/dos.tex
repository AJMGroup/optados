\documentclass[a4paper,11pt,twoside]{book}
%\documentclass[11pt,twoside,letterpaper]{book}
\usepackage{a4wide}
%\usepackage{multirow}
\usepackage{footnote}
\usepackage{amsbsy}
\usepackage[dvips]{graphicx}
%\usepackage{fancyheadings}
\usepackage{fancyhdr}
%\setlength{\parindent}{0in}
%\setlength{\parskip}{0.05in}
\setlength{\parskip}{0.1in}

%\usepackage{a4wide}
\usepackage{color}
\def\red{\textcolor{red}}

%\parskip=2mm

% Ensure that blank pages don't have numbers or heading on them 
\makeatletter
\def\cleardoublepage{\clearpage\if@twoside \ifodd\c@page\else
 \hbox{}
 \vspace*{\fill}
 \thispagestyle{empty}
 \newpage\fi\fi}
\makeatother

% set fancy headings
\pagestyle{fancy}
\lhead[{\it \thepage}]{{\bf\it {\tt OptaDOS}: User Guide}}
\chead{}
\rhead[{\bf\it {\tt OptaDOS}: User Guide}]{{\it \thepage}}
\renewcommand{\headrulewidth}{0.2pt}
\lfoot{}
\cfoot{}
\rfoot{}
\renewcommand{\footrulewidth}{0pt}
\setlength{\footskip}{0.25in}
\setlength{\parindent}{0in}

\title{{\huge {\tt OptaDOS}: User Guide}\\ {Version 1.0}}

\author{Andrew J. Morris, Rebecca J. Nicholls, Chris J. Pickard, Jonathan R. Yates \\
\\
Department of Materials\\
University of Oxford\\
Parks Road\\
Oxford OX1 3PH\\
UK \\
\\
{\small and} \\
\\
Department of Physics and Astronomy\\
Univeristy College London\\
Gower Street\\
London, WC1E 6BT\\
UK}


\date{May 2011}

\begin{document}
\newcommand{\optados}{\texttt{OptaDOS}}
\newcommand{\lindos}{\texttt{LinDOS}}
\newcommand{\castep}{\textsc{castep}}
\maketitle

\setcounter{tocdepth}{1}
\tableofcontents

\chapter{Introduction}\label{chap:introduction}
\section{Background}
\optados\ is a code for calculating optical, core-level excitation spectra along with full, partial and joint electronic density of states (DOS).  The code was developed by merging the \lindos\ code of Andrew Morris and Chris Pickard at University College London with the optical properties code of Rebecca Nicholls and Jonathan Yates at Oxford University.  \optados\ is written in Fortran 95 and may be run in parallel using {\tt MPI}.  At present \optados\ interfaces with \castep\ output files, although it is extendible to perform calculations on any set of band eigenvalues and their derivatives generated by any electronic structure code.

The code is freely available through the GPL licence with the request that the following citation (quoted in full) is required in any publication resulting from the use of \optados.

\emph{Andrew\,J. Morris, R.\,J. Nicholls, C.\,J. Pickard and Jonathan\,R. Yates, The \optados\ code, Comp. Phys. Comm. (2011)}.

\begin{center}
Further information and examples can be found at

\verb#www.optados.org#
\end{center}


\section{Features}
\optados\ generates optical, core-level excitation spectra along with full, partial and joint electronic DOS. The DOS, PDOS and JDOS take advantage of the linear and adaptive smearing schemes which are more accurate than standard Gaussian smearing since they exploit knowledge of the gradients of the bands at each k-point in the Brillouin zone.  These DOS are the basis of the more advanced functionality of \optados, the core and optical spectra.

Along with data text files \optados\ also generates \verb#.agr# files of results to be read by \verb#grace#. 


\chapter{Getting Started}\label{chap:getting_started}
\section{Installation}
\optados\ is usually obtained in a gzipped tarball, \verb#optados-X.X.tar.gz#. Extract this ( \verb# tar -xzf optados-X.X.tar.gz#) in the desired directory. Inside the  \verb#optados/# directory are a number of sub directories,  \verb#documents/#,   \verb#examples/#.  The code may be compiled using the \verb#Makefile# in the  \verb#optados/# directory.  The \verb#SYSTEM#, \verb#BUILD#, \verb#COMMS_ARCH# and \verb#PREFIX# flags must be set, either in the  \verb#Makefile#, or from the command line (for example  \verb#make BUILD=fast#).  

\subsection[system]{\tt SYSTEM}

Choose which compiler to use to make \optados. The valid values are:
\begin{itemize}
\item[{\bf --}]  \verb#g95# (default)
\item[{\bf --}]  \verb#gfortran#
\item[{\bf --}]  \verb#ifort#
\item[{\bf --}]  \verb#nag#
\item[{\bf --}]  \verb#pathscale#
\item[{\bf --}]  \verb#pgf90#
\item[{\bf --}]  \verb#sun#
\end{itemize}

\subsection[build]{\tt BUILD}

Choose the level of optimisations required when making \optados.  The valid values are:
\begin{itemize}
\item[{\bf --}]  \verb#fast# (default) All optimisations
\item[{\bf --}]  \verb#debug# No optimisations, full debug information
\end{itemize}

\subsection[comms_arch]{\tt COMMS\_ARCH}

Whether to compile for serial or parallel execution. The valid values are:
\begin{itemize}
\item[{\bf --}]  \verb#serial# (default) 
\item[{\bf --}]  \verb#mpi# 
\end{itemize}

\subsection[bin_dir]{\tt PREFIX}
Choose where to place the \optados\ binary. The default is the \optados\ directory.


\section{Usage}
{\tt
\begin{quote}
optados.x86\_64  [seedname]
\end{quote} }
\begin{itemize}
\item{  {\tt seedname}: If a seedname string is given the code will read its input from a file {\tt seedname.odi}. The default value is \castep.}
\end{itemize}

\chapter{Parameters}\label{chap:parameters}

\section{{\tt seedname.odi} File}
The \optados\ input file {\tt seedname.odi} has a flexible free-form
structure. 

The ordering of the keywords is not significant. Case is ignored (so
\verb#smearing_width# is the same as \verb#Smearing_Width#). Characters after !, or \#
are treated as comments. Most keywords have a default value that is
used unless the keyword is given in {\tt seedname.odi}. Keywords may be set
in any of the following ways
{\tt
\begin{quote}
smearing\_width = 0.4

smearing\_width : 0.4

smearing\_width   0.4
\end{quote} }
A logical keyword can be set to {\tt .true.} using any of the following
strings: {\tt T}, {\tt true}, {\tt .true.}.


\clearpage


\section{Parameters}
\subsection[task]{\tt character(lex=20) :: task}

Tells the code what to compute. 

The valid options for this parameter are:
\begin{itemize}
\item[{\bf --}]  \verb#dos# (default)
\item[{\bf --}]  \verb#compare_dos#
\item[{\bf --}]  \verb#compare_jdos#
\item[{\bf --}]  \verb#jdos#
\item[{\bf --}]  \verb#pdos#
\item[{\bf --}]  \verb#optics#
\item[{\bf --}]  \verb#core#
\item[{\bf --}]  \verb#all#
\end{itemize}
Several tasks can be specified \emph{e.g.} to compute doi and jdos use
\verb#task : dos jdos#.  
However, the \verb#compare_dos# and \verb#compare_jdos# tasks can only be combined with each other, and no additional tasks.
\verb#compare_dos# and \verb#compare_jdos# calculate the DOS and JDOS respectively using all broadening schemes. It is good practice to check the quality of the underlying DOS before other tasks are requested. 

\subsection[broadening]{\tt character(lex=50) :: broadening}

Specified the scheme used to broaden a discrete sampling of the
Brillouin Zone to a continuous spectral function.

The valid options for this parameter are:
\begin{itemize}
\item[{\bf --}]  \verb#adaptive# (default)
\item[{\bf --}]  \verb#fixed#
\item[{\bf --}]  \verb#linear#
\item[{\bf --}]  \verb#quad# (not currently implemented)
\end{itemize}

\subsection[iprint]{\tt integer :: iprint}

This indicates the level of verbosity of the output from 1,
the bare minimum: 2, with progress reports: to 3, which corresponds to full debugging output.

The default value is 1.


\subsection[energy\_unit]{\tt character(Len=20) :: energy\_unit}
The energy unit to be used for writing quantities in the output files.

The valid options for this parameter are:
\begin{itemize}
\item[{\bf --}]  \verb#eV# (default)
\item[{\bf --}]  \verb#Ry#
\item[{\bf --}]  \verb#Ha#
\end{itemize}

\subsection[adaptive\_smearing]{\tt logical :: legacy\_file\_format}
\begin{itemize}
\item[{\bf --}] \verb#TRUE#  Read \castep\ input compatible with versions $<$ 6.0.
\item[{\bf --}] \verb#FALSE# (Default) Read \castep\ input compatible for use with \castep\ versions 6.0$+$ and generated with the castep spectral task.
\end{itemize}

\subsection[adaptive\_smearing]{\tt real(kind=dp) :: adaptive\_smearing}
Set the relative smearing in the adaptive scheme.

Default value is 1.4

\subsection[fixed\_smearing]{\tt real(kind=dp) :: fixed\_smearing}
Smearing width for fixed broadening.

If $\verb#spectral_scheme# = \verb#fixed#$ default value is 0.3eV.

\subsection[scissor\_op]{\tt real(kind=dp) :: scissor\_op}
Value of the scissor operator. 

Default value is 0\,eV (\emph{i.e.} not used)

\subsection[compute\_efermi]{{\tt logical :: compute\_efermi}}

If {\tt compute\_efermi=TRUE}, then \optados\ will use the value of the Fermi
level computed from the integration of the DOS. If {\tt
  compute\_efermi=FALSE}, then the value set by {\tt fermi\_energy} will be
used, if {\tt fermi\_energy} is not set the value from the bands file will be used.

The default value is {\tt FALSE}.

\subsection[fermi\_energy]{\tt real(kind=dp) :: fermi\_energy}
Value of the Fermi energy.

No default value

\subsection[output\_format]{\tt character(len=20) :: output\_format}
Format in which to output data.

The valid options for this parameter are:
\begin{itemize}
\item[{\bf --}]  \verb#gnuplot# (not implemented yet)
\item[{\bf --}]  \verb#grace# (default)
\end{itemize}

\subsection[finite\_bin\_correction]{\tt logical :: finite\_bin\_correction}
Force each Gaussian to be larger than a single energy bin. (Useful for adaptive smearing and semi-core states when \verb#numerical_intdos=TRUE#). 

Default value \verb#FALSE#.

\subsection[numerical\_intdos]{\tt logical :: numerical\_intdos}
Calculate the integrated dos by numerical integration instead of semi-analytically. (Useful for comparison with \lindos.)

Default value \verb#FALSE#.

\subsection[hybrid\_linear]{\tt logical :: hybrid\_linear}
Switch from linear broadening scheme to adaptive broadening when band gradient less than \verb#hybrid_linear_grad_tol#. 
This allows for a good description of very flat bands such as defect and semi-core states. May also be used in conjunction 
with \verb#finite_bin_correction# further improving the DOS and band energy

Default value \verb#FALSE#.

\subsection[hybrid\_linear\_grad\_tol]{\tt real(kind=dp) :: hybrid\_linear\_grad\_tol}
Tolrance for switching from linear to adaptive broadening when using hybrid\_linear option.

The default value is 0.01eV/\AA.


\subsection[set\_efermi\_zero]{\tt logical :: set\_efermi\_zero}
Shift energy scales so that the Fermi energy is at 0.

Default value \verb#TRUE#.

\subsection[compute\_band\_energy]{\tt logical :: compute\_band\_energy}
Compute the band energy by summing bands both using \castep{'s} eigenvalue and \optados{'s} density of states.

Default value \verb#TRUE#.

\subsection[DSO\_per\_volume]{\tt logical :: dos\_per\_volume}
Present DOS per simulation cell volume

Default value \verb#FALSE#.

\subsection[dos\_min\_energy]{\tt real(kind=dp) :: dos\_min\_energy}
Lower energy range for DOS and related properties.

Default value is 5eV below the lowest eigenvalue in the bands file.

\subsection[dos\_max\_energy]{\tt real(kind=dp) :: dos\_max\_energy}
Upper energy range for DOS and related properties.

Default value is 5eV above the highest eigenvalue in the bands file.


\subsection[dos\_nbins]{\tt real(kind=dp) :: dos\_nbins}
Instead of setting a Default value \verb#dos_spacing# the total number of DOS bins may be given.  (Useful for comparison with \lindos.)

\subsection[dos\_spacing]{\tt real(kind=dp) :: dos\_spacing}
Resolution at which to compute the DOS and related properties.
Default value is 0.1eV 


\subsection[jdos\_max\_energy]{\tt real(kind=dp) :: jdos\_max\_energy}
Upper energy range for JDOS and related properties.

Default value is the difference between the valence band maximum (or
Fermi level) and the highest eigenvalue in the bands file.

\subsection[jdos\_spacing]{\tt real(kind=dp) :: jdos\_spacing} 
Resolution at which to compute the DOS and related properties.
Default value is 0.01eV. 

\subsection[pdos]{\tt character :: pdos}
Defines which components to include in the pdos analysis:

\begin{itemize}
\item[{\bf --}]  \verb#angular# (decompose as s,p,d \emph{etc.})
\item[{\bf --}]  \verb#sites#    (decompose onto atomic sites, C, H \emph{etc.})
\item[{\bf --}]  \verb#species#    (decompose onto atomic species C1, H1, H2 \emph{etc.})
\item[{\bf --}]  \verb#species_ang#    (decompose onto angular momentum channels and species C1s, C1p \emph{etc.})
\item[{\bf --}]  \verb#C:H#     (decompose onto Carbon and Hydrogen sites)
\item[{\bf --}]  \verb#C1:C3:C4-C8#  (decompose onto atoms C1, C2 and C4,C5,C6,C7,C8)
\item[{\bf --}]  \verb#Si1[s;d]#     (decompose onto 's' and 'd' channels for
  atom Si1)
\item[{\bf --}]  \verb#sum:C1:C3:C4-C8#  (decompose onto atoms C1, C2 and C4,C5,C6,C7,C8 and combine into the single projection)

\end{itemize}


\subsection[optics\_geom]{\tt character(len=20) :: optics\_geom}

Specifies the geometry for the optics calculation.  Possible options:
\begin{itemize}
\item[{\bf --}]  \verb#polycrystalline# (Isotropic average)
\item[{\bf --}]  \verb#polarized#  
\item[{\bf --}]  \verb#unpolarized# 
\item[{\bf --}]  \verb#tensor# (Full dielectric tensor)
\end{itemize}
The default is polycrystalline.

\subsection[optics\_qdir]{\tt real(kind=dp) :: optics\_qdir(3)}
Direction of polarisation. Must be specified if \verb#optics_geom :polarized#  
or \verb#optics_geom : unpolarized#.
No default

\subsection[optics\_intraband]{\tt logical :: optics\_intraband}
Calculate the intraband contribution to the dielectric function.  (Important for metals.)    
The default is FALSE.

\subsection[optics\_drude\_broadening]{\tt real(kind=dp) :: optics\_drude\_broadening}
Value of broadening included in the Drude term expressed in $s^{-1}$.  
The default value is 1E-14.  

\subsection[core\_geom]{\tt character(len=20) :: core\_geom}

Specifies the geometry for the core-spectra calculation.  Possible options: 
\begin{itemize}
\item[{\bf --}]  \verb#polycrystalline# (Isotropic average)
\item[{\bf --}]  \verb#polarized#  
\end{itemize}
The default is polycrystalline.

\subsection[core\_qdir]{\tt real(kind=dp) :: core\_qdir(3)}
Direction of polarisation. Must be specified if \verb#core_geom :polarized#.
No default

\subsection[core\_LAI\_broadening]{\tt logical :: core\_LAI\_broadening}
Include life-time and instrumentation broadening.  
The default is FALSE.

\subsection[core\_gaussian\_width]{\tt real(kind=dp) :: LAI\_gaussian\_width}
FWHM of Gaussian function used to broaden spectrum.  

The default value, if \verb#core_LAI_broadening :true#, is 0 (i.e. no Gaussian used).

\subsection[core\_lorentzian\_width]{\tt real(kind=dp) :: LAI\_lorentzian\_width}
FWHM of fixed Lorentzian function used to broaden spectrum.  

The default value, if \verb#core_LAI_broadening :true#, is 0 (i.e. no fixed Lorentzian used).

\subsection[core\_lorentzian\_scale]{\tt real(kind=dp) :: LAI\_lorentzian\_scale}
Variation of Lorentzian function with energy i.e. the width of the Lorentzian is energy x 
\verb#core_lorentzian_scale#.  If set to zero, no energy dependent broadening is included.  
If \verb#core_lorentzian_scale# and \verb#core_lorentzian_width# are both specified, the 
total width of the Lorentzian used will be \verb#core_lorentzian_width# + (energy x 
\verb#core_lorentzian_scale#).

The default value, if \verb#core_LAI_broadening :true#, is 0.1. 

\subsection[core\_lorentzian\_offset]{\tt real(kind=dp) :: LAI\_lorentzian\_offset}
Energy (in eV) above the edge onset that the energy dependent broadening starts.  
The default value, if \verb#core_LAI_broadening :true#, is 0.



\subsection[devel\_flag]{\tt character(len=50) :: devel\_flag}

Not a regular keyword. Its purpose is to allow a developer to pass a
string into the code to be used inside a new routine as it is developed.

No default.



\chapter{Examples}

Each example in the \verb#examples/# directory contains example \castep\ input files and a sample \optados\ input file.
\section{DOS}
See  \verb#examples/Si2_DOS/#.  This is a simple example of using \optados\ for calculating electronic density of states.  We choose to recalculate the Fermi level using the calculated DOS, rather than use the Fermi level suggested by \castep. The DOS is outputted to {\tt Si2.adaptive.dat}. A file suitable for plotting using {\tt xmgrace} is written to {\tt Si2.adaptive.agr}.  If {\tt TASK : compare\_dos} is used instead, \optados\ will calculate DOS using all the broadening methods, this is good practice to see whether the broadening widths are appropriate before more advanced tasks are carried out, such as JDOS, core and optical calculations.

\section{PDOS}
See  \verb#examples/Si2_PDOS/#. This is a simple example of using \optados\ for calculating partial electronic density of states. We choose to decompose the DOS into angular momentum channels ({\tt PDOS : angular}). We choose to recalculate the Fermi level using the calculated DOS, rather than use the Fermi level suggested by \castep.  The output can be found in {\tt Si2.pdos.dat}. Other things to try are:
\begin{itemize}
\item[{\bf --}]  \verb#PDOS : Si1;Si2(s)#  -- Output the PDOS on Si atom 1 and the PDOS on the s-channel of Si atom 2. (Resulting in two projectors)
\item[{\bf --}]  \verb#PDOS : sum:Si1-2(s)#  --  Output the sum of the s-channels on the two Si atoms. (Resulting in one projector)
\item[{\bf --}]  \verb#PDOS : Si1(p)# -- Output the p-channel on Si atom 1. (Resulting in one projector)
\end{itemize}



\section{JDOS}
See  \verb#examples/Si2_JDOS/#. This is a simple example of using \optados\ for calculating joint electronic density of states. This is a simple example of using \optados\ for calculating joint electronic density of states.  We choose to recalculate the Fermi level using the calculated DOS, rather than use the Fermi level suggested by \castep.  If {\tt TASK : compare\_jdos} is used instead, \optados\ will calculate the JDOS using all the broadening methods, this is good practice to see whether the broadening widths are appropriate before more advanced tasks are carried out.  The JDOS is outputted to {\tt Si2.jadaptive.dat}. A file suitable for plotting using {\tt xmgrace} is written to {\tt Si2.jadaptive.agr}. 

\section{CORE}
See  \verb#examples/diamond_CORE/#. This is a simple example of using \optados\ for calculating core level spectra. 

\section{OPTICS}
See  \verb#examples/Al_OPTICS/#. This is a simple example of using \optados\ for calculating optical properties. 

\end{document}

% LocalWords:  odi jdos dos
